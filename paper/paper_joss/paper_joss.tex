\documentclass[
]{jss}

\usepackage[utf8]{inputenc}

\providecommand{\tightlist}{%
  \setlength{\itemsep}{0pt}\setlength{\parskip}{0pt}}

\author{
Lauren Kennedy\\Monash University \And Mitzi Morris\\Columbia
University \And Jonah Gabry\\Columbia University \And Rohan
Alexander\\University of Toronto
}
\title{\pkg{MRP-Kit} A `grammar' of multilevel regression with
post-stratification and implementation in \proglang{R}}

\Plainauthor{Lauren Kennedy, Mitzi Morris, Jonah Gabry, Rohan Alexander}
\Plaintitle{A `grammar' of multilevel regression with
post-stratification and implementation in R}
\Shorttitle{\pkg{MRP-Kit}: A grammar of MRP}

\Abstract{
In this paper we define a `grammar' for multilevel regression with
post-stratification (MRP) and implement this grammar in the R Package
\pkg{MRPKit}. Our grammar is centered around the following verbs: add,
remove, delete, and replace. These act on survey objects. Our grammar,
and its implementation, provides a detailed workflow when conducting MRP
that will be useful for researchers and in teaching.
}

\Keywords{multilevel regression with
post-stratification, \proglang{R}, reproducibility, statistics, political
science}
\Plainkeywords{multilevel regression with
post-stratification, R, reproducibility, statistics, political science}

%% publication information
%% \Volume{50}
%% \Issue{9}
%% \Month{June}
%% \Year{2012}
%% \Submitdate{}
%% \Acceptdate{2012-06-04}

\Address{
        }

% Pandoc citation processing

% Pandoc header

\usepackage{amsmath}

\begin{document}

\section[Introduction]{Introduction} \label{sec:intro}

Multilevel regression with post-stratification (MRP) is a statistical
approach in which surveys are related to each other using a statistical
model. This is important because known biases in one survey, can be
adjusted for by another in a statistically reasonable way. This enables
better use of non-representative surveys, additional information, and
propagation of uncertainty. However it can be difficult to use MRP due
to this need to related two different datasets. This package defines a
grammar, or list of underlying rules, of MRP and then describes an R
package, \proglang{MRPKit}, that implements this grammar.

At its core, MRP is a mapping between a survey object and a population
object. It is from this mapping that the power of MRP exists, but it
also establishing this mapping that is the difficult part of
implementing MRP models. Making this implementation easier and more
reproducible is important as interest in, and the use of, MRP increases.
It can be difficult even for those experienced with MRP to ensure there
are no mistakes in this mapping and easing this is an important
contribution to enhancing the reproducibility of MRP analysis.

We first define a grammar of MRP, which we define as the underlying
rules and principles that are common to every analysis based on MRP.
This grammar is based around the following verbs: `add', `remove',
`delete', and `replace'. These verbs are applied to a survey object, a
post-stratification object, and survey\_map object. These three objects
come together to create a mapping object, which is what processes such
as regression act on. Finally, common diagnostics and graphs are
implemented. In this way, our grammar and package implement an entire
statistical workflow for conducting MRP.

The survey object would typically be a regular survey, such as a
political poll of 1,000 respondents, but it could also be a larger
survey, such as the Canadian Election Survey, or similar. The
post-stratification object would typically be a larger survey, such as,
in the case of the US, the (INSERT THE USUAL ONE), or a census.

Our grammar and package complements existing packages such as
\proglang{survey} \citep[\citet{lumleytwo},
\citet{lumleythree}]{lumleyone}, and \proglang{DeclareDesign}
\citep{citeDeclareDesign}. These packages are focused on the designing,
implementing, and simulating from, surveys. Instead, ours is focused on
what is needed for MRP.

The remainder of this paper is structured as follows: Section
\ref{sec:review} reviews similar packages and contributions and places
ours within that context. Section \ref{sec:components} discusses the
grammar and the core aspects of MRP as implemented in \proglang{MRPKit}.
Section \ref{sec:implementation} discusses some of the implementation
issues and technical notes related to the decisions that were made.
Section \ref{sec:vignette} provides two examples of the package in use,
one using SOMETHING, and the other using SOMETHING ELSE. Finally,
Section \ref{sec:summary} provides a summary discussion, some cautions
and weaknesses, as well as notes about next steps.

\section{Review of the other packages} \label{sec:review}

The most common alternative at the moment to this package is for users
to do all aspects themselves. While there is nothing inherently
difficult about MRP, the implementation can be difficult. In particular,
preparing and matching different levels between surveys can be time
consuming and potentially introduce undocumented errors.

The \proglang{survey} package \citep[\citet{lumleytwo},
\citet{lumleythree}]{lumleyone} \ldots{}

The \proglang{DeclareDesign} package \citep{citeDeclareDesign}\ldots{}

There is also an existing package called \proglang{MRP} available here:
\url{https://github.com/gelman/mrp}. However it does not appear to have
been updated in some time.

The main alternative approach is for MRP to be conducted on a
case-specific basis. That is a researcher interested in MRP estimates,
writes the code needed to obtain, clean, prepare, analyze and ultimately
interpret the estimates. Again, while there is nothing inherently wrong
with this approach, successfully conducting MRP requires dealing with a
large number of small issues. Each of these is small in their own right,
but getting them wrong can have large effects that are potentially
unnoticed on the estimates.

\section{Components and grammar} \label{sec:components}

The \proglang{MRPKit} package has the following key components: survey
objects, which for most users will be a collection of two surveys where
one is larger than the other; a survey map, that relates the survey
objects and then an MRP object, which is created once the survey map is
applied to the survey objects. Regression acts on the MRP object. These
objects are subject to the following verbs: `add', `delete', `new', and
`replace'. For instance: \proglang{SurveyMap\$add},
\proglang{SurveyOb\$delete}, \proglang{SurveyMap\$new}, and
\proglang{SurveyMap\$replace}.

\subsection*{Survey objects}

Survey objects are the surveys that MRP will bring together. A user can
\proglang{add} a survey object by providing a CSV file location. The
user then needs to identify the data types of each column. Typically,
there will need to be two survey objects, where one will be the survey
that is of interest for its response variables, such as political
opinion, and another will be a survey that is to be used for
post-stratification.

New is when you have a new survey object and you are bringing it in.
e.g.

What is the difference between add and new? What is replace doing? URGH,
I just need to look at the repo.

\subsection*{Survey map}

A \texttt{SurveyMap} object holds the mapping between a set of items in
a survey and a population dataset. The label is the item label in each
dataset and the values is a list of all possible values. The values for
the survey and population must be aligned, i.e., the lists must have the
same number of elements and the values at index i in each list are
equivalent. If there is a meaningful ordering over the values, they
should be listed in that order, either descending or ascending.

One of the fundamental issues when conducting MRP is to ensure that

\subsection*{MRP object}

An MRP object contains survey objects, and a survey map. At the point at
which the survey objects are put into the MRP object they become
immutable but the survey map object is.

The MRP object outputs an analysis. There would be a MRP model fit class
that would

Many objects on the same class. There are two

\section{Implementation and technical notes} \label{sec:implementation}

\section{Vignette} \label{sec:vignette}

\subsection*{Simulated data}

\subsection*{Using CCES data}

\subsection*{Using your own data}

\section{Summary and discussion} \label{sec:summary}

\subsection*{Next steps and cautions}

\section*{Acknowledgments}

\bibliography{references.bib}


\end{document}
