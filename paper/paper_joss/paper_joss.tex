\documentclass[
]{jss}

\usepackage[utf8]{inputenc}

\providecommand{\tightlist}{%
  \setlength{\itemsep}{0pt}\setlength{\parskip}{0pt}}

\author{
Lauren Kennedy\\Monash University \And Mitzi Morris\\Columbia
University \And Jonah Gabry\\Columbia University \And Rohan
Alexander\\University of Toronto \And Dewi Amaliah\\Monash University
}
\title{\pkg{MRP-Kit} A `grammar' of multilevel regression with
post-stratification and implementation in \proglang{R}}

\Plainauthor{Lauren Kennedy, Mitzi Morris, Jonah Gabry, Rohan
Alexander, Dewi Amaliah}
\Plaintitle{A `grammar' of multilevel regression with
post-stratification and implementation in R}
\Shorttitle{\pkg{MRP-Kit}: A grammar of MRP}

\Abstract{
In this paper we define a `grammar' for multilevel regression with
post-stratification (MRP) and implement this grammar in the R Package
\pkg{MRPKit}. Our grammar is centered around the following verbs: new,
validate, add, delete, replace, mapping, tabulate, fit, predictify,
collapsify, visify. These act on survey objects. Our grammar, and its
implementation, provides a detailed workflow when conducting MRP that
will be useful for researchers and in teaching.
}

\Keywords{multilevel regression with
post-stratification, \proglang{R}, reproducibility, statistics}
\Plainkeywords{multilevel regression with
post-stratification, R, reproducibility, statistics}

%% publication information
%% \Volume{50}
%% \Issue{9}
%% \Month{June}
%% \Year{2012}
%% \Submitdate{}
%% \Acceptdate{2012-06-04}

\Address{
          }

% Pandoc citation processing

% Pandoc header

\usepackage{amsmath}

\begin{document}

\section[Introduction]{Introduction} \label{sec:intro}

Multilevel regression with post-stratification (MRP) is a statistical
approach in which surveys are related to each other using a statistical
model. This is important because known biases in one survey can be
adjusted for by another in a statistically reasonable way. This enables
better use of non-representative surveys, additional information, and
propagation of uncertainty. However, it can be difficult to use MRP due
to this need to relate (at least) two different datasets. This package
defines a grammar, or list of underlying rules, for MRP and then
describes an R package, \proglang{MRPKit}, that implements this grammar.

At its core, MRP is a mapping between a survey object and a population
object. It is from this mapping that the power of MRP exists, but it can
be difficult to establish this mapping in a reproducible way that fits
into a typical applied statistics workflow. Making this implementation
easier and more reproducible is important as interest in, and the use
of, MRP increases. It can be difficult even for those experienced with
MRP to ensure there are no mistakes in this mapping and creating a
well-defined workflow for MRP is an important contribution to enhancing
the reproducibility of MRP analysis.

We first define a grammar of MRP, which we define as the underlying
rules and principles that are common to every analysis based on MRP.
This grammar is based around the following verbs: `new', `validate',
`add', `delete', `replace', `mapping', `tabulate', `fit', `predictify',
`collapsify', and `visify'. These verbs are applied to a survey object,
a post-stratification object, and survey-map object. These three objects
come together to create a mapping object, which is what estimation
processes such as regression act on. Finally, common diagnostics and
graphs are implemented. In this way, our grammar and package implement
an entire statistical workflow for MRP.

The survey object would typically be a regular survey, such as a
political poll of 1,000 respondents, but it could also be a larger
survey, such as the Canadian Election Survey, or similar. The
post-stratification object would typically be a larger survey, such as,
in the case of the US, the American Community Surveys, or a census.

Our grammar and package complements existing packages such as
\proglang{survey} \citep[\citet{lumleytwo},
\citet{lumleythree}]{lumleyone}, and \proglang{DeclareDesign}
\citep{citeDeclareDesign}. These packages are focused on the designing,
implementing, and simulating from, surveys. Instead, ours is focused on
what is needed for MRP, and we draw on their packages where possible.

The remainder of this paper is structured as follows: Section
\ref{sec:review} reviews similar packages and contributions and places
ours within that context. Section \ref{sec:components} discusses the
grammar and the core aspects of MRP as implemented in \proglang{MRPKit}.
Section \ref{sec:implementation} discusses some of the implementation
issues and technical notes related to the decisions that were made.
Section \ref{sec:vignette} provides two examples of the package in use,
one using SOMETHING, and the other using SOMETHING ELSE. Finally,
Section \ref{sec:summary} provides a summary discussion, some cautions
and weaknesses, as well as notes about next steps.

\section{Review of related packages} \label{sec:review}

The most common alternative at the moment to this package is for users
to do all aspects of the MRP workflow themselves. While there is nothing
inherently difficult about MRP, the implementation can be difficult. In
particular, preparing and matching different levels between surveys can
be time consuming and potentially introduce undocumented errors.

The \proglang{survey} package \citep[\citet{lumleytwo},
\citet{lumleythree}]{lumleyone} is well-established and allows a user to
specify a survey design, such as clustered, stratified, etc, generate
summary statistics, and conduct sub-population analysis. The
\proglang{DeclareDesign} package \citep{citeDeclareDesign} is a newer
package that is based around a grammar of research design aspects
including: data generation, creating treatment and control groups, and
sampling. There is also an existing package called \proglang{MRP}
available here: \url{https://github.com/gelman/mrp}. However, it does
not appear to have been updated in some time.

The main alternative approach is for MRP to be conducted on a
case-specific basis. This means that a researcher who is interested in
MRP estimates writes the code needed to obtain, clean, prepare, analyze
and ultimately interpret the estimates. Again, while there is nothing
inherently wrong with this approach, successfully conducting MRP
requires dealing with a large number of small issues. Each of these is
small in their own right but getting them wrong can have large effects
that are potentially unnoticed on the estimates. It is also easy to
overlook steps, or to not appropriately document them. The
\proglang{MRPKit} package is prescriptive about the steps that must be
taken, and ensures that all aspects are documented.

\section{Components and grammar} \label{sec:components}

The \proglang{MRPKit} package has the following key components:
\proglang{SurveyData} objects, which for most users will be a collection
of two surveys where one is larger than the other; a
\proglang{SurveyQuestion} object, that holds the mapping for one
question or demographic between the survey and post-stratification
dataset; and a \proglang{SurveyMap} object, which holds the mapping
between a set of items in a survey and a post-stratification dataset.
These are expanded on in the following sub-sections before being brought
together.

\subsection{SurveyData objects}

A \proglang{SurveyData} object represents a survey and its metadata. The
survey itself is a data frame and importantly has a series of metadata.
The survey metadata consists of the following:

\begin{itemize}
\tightlist
\item
  per-column questions: a list of strings
\item
  per-column allowed response values: a list of character vectors
\item
  per-column survey weights: a vector of numeric weights
\item
  survey design: a string that specifies the survey design using
  \proglang{lme4} style formula syntax.
\end{itemize}

Some examples of survey designs include:

\begin{itemize}
\tightlist
\item
  \texttt{\textasciitilde{}.}: a random sample
\item
  \texttt{\textasciitilde{}\ (1\textbar{}cluster)}: a one stage cluster
  sample
\item
  \texttt{\textasciitilde{}\ stratum}: a stratified sample
\end{itemize}

These \proglang{SurveyData} objects are the surveys that MRP will bring
together. A user can use the verb \texttt{add} to create a survey object
by providing a CSV file location. The user then needs to identify the
data types of each column. Typically, there will need to be two survey
objects, where one will be the survey that is of interest for its
response variables, such as political opinion, and another will be a
survey that is to be used for post-stratification.

\subsection{SurveyQuestion objects}

A \proglang{SurveyQuestion} object holds the mapping for one question or
demographic between the survey and post-stratification dataset. For
instance, if both the survey and post-stratification dataset contain a
question about age, it may be that answers are expressed as a single
integer representing years and so mapping them is relatively
straight-forward. This is done with the verb \texttt{new}, for instance:

\begin{CodeChunk}
\begin{CodeInput}
R> SurveyQuestion$new(
+   name = "age",
+   col_names = c("age1","age2"),
+   values_map = list(
+     "18" = "18", 
+     "19" = "19",
+     "20" = "20"
+     )
+ )
\end{CodeInput}
\end{CodeChunk}

If ages have been aggregating into age-groups then it may not be the
case that both the survey and the post-stratification dataset have
groups that directly correspond. In that case, a user needs to specify
how the age-groups fit together. This is also done with the verb
\texttt{new}, for instance:

\begin{CodeChunk}
\begin{CodeInput}
R> SurveyQuestion$new(
+   name = "age",
+   col_names = c("age1","age2"),
+   values_map = list(
+     "18-25" = "18-35", 
+     "26-35" = "18-35",
+     "36-45" = "36-55",
+     "46-55" = "36-55",
+     "56-65" = "56-65",
+     "66-75" = "66+",
+     "76-90" = "66+"
+     )
+ )
\end{CodeInput}
\end{CodeChunk}

In this example, the age-groups do not neatly match, however they can be
made to match. For instance, the 36-45 and 46-55 age-groups in the
survey correspond to the 36-55 age-group in the post-stratification
dataset.

If there is not a clean correspondence between the survey and
post-stratification datasets, then the user needs to decide which
age-group to assign to.

\subsection{SurveyMap objects}

A \proglang{SurveyMap} object holds the mapping between a set of items
in a survey and a population dataset. That is, it draws on two
\proglang{SurveyData} objects, and a \proglang{SurveyQuestion} object,
and brings the two \proglang{SurveyData} objects together following the
correspondences provided in the \proglang{SurveyQuestion} object.

The label is the item label in each dataset and the values is a list of
all possible values. The values for the survey and post-stratification
must be aligned, i.e., the lists must have the same number of elements
and the values at index i in each list are equivalent. If there is a
meaningful ordering over the values, they should be listed in that
order, either descending or ascending.

One of the fundamental issues when conducting MRP is to ensure that the
survey and post-stratification datasets are aligned. This object ensures
that is the case before a user conducts their analysis.

For instance, if a user decides to use age-group and gender, then the
map would specify the correspondences in age-group and also in gender.
If they then also decided to use education, then the map would also
specify that correspondence. They would then use the verb \texttt{add}
to include that third question in the map.

The verb \texttt{validate} is used on \proglang{SurveyMap} objects to
assess whether the proposed map is appropriate. This means that an error
occurs if there is a mismatch, and a warning occurs if there are
variables that are available in either the survey or post-stratification
datasets that have not been mapped and hence would be unavailable to
model.

After the \proglang{SurveyMap} object has been validated then the verb
\texttt{mapping} is used to prepare it to be modelled.

A model is fit to a \proglang{SurveyMap} object using the verb
\texttt{fit}. After a model has been fit it can be used to get MRP-based
estimates using \texttt{predictify()}. And finally the estimates can be
visualized using \texttt{visify()}.

\subsection{Putting them together}

We provide examples of implementation in Section \ref{sec:vignette},
however, we want to explain putting the three objects together here and
to briefly bring the three objects together.

To begin with, one needs two datasets. These would be created using
\texttt{SurveyObj\$new}, specifying the questions, responses, weights,
and design.

Having established both a survey dataset and a post-stratification
dataset, a user need to specify how the questions in the survey dataset
correspond to the questions in the post-stratification dataset. This is
done on a question-by-question basis using \texttt{question\$new}
assigning a name for that question, the labels that are used in the
survey and post-stratification datasets, and the values. A user does
this in such a way that the values are matched between the two datasets.

The user now need to put these questions together and does this using
\texttt{SurveyMap\$new} and specifying the survey object, the
post-stratification object, and the questions are to be used. A user can
validate this map with \texttt{validate}, and potentially add additional
questions using \texttt{add}, delete some existing ones using
\texttt{delete}, or potentially replace a question using
\texttt{replace}. Once a user is happy with the correspondence to be
used they can establish a mapping using \texttt{mapping}.

The relationship can then be modelled using \texttt{fit}. There are a
variety of options that are available as built-into the package,
including \texttt{rstanarm::stan\_glmer}, and \texttt{brms::brm}. A user
could also specify their own modelling approach if they wanted to. In
any case, once a model has been fit to the survey dataset, it can be
used to create estimates based on the post-stratification dataset with
\texttt{predictify}, which takes a fitted model as an argument.
Typically a user is interested in some type of high-level estimates, for
instance on a province/state basis, and \texttt{collapsify} provides the
functionality to do this. Finally, the estimates on that collapsed basis
can be visualized with \texttt{visify}

\section{Implementation and technical notes} \label{sec:implementation}

\section{Vignette} \label{sec:vignette}

\subsection*{Simulated data}

\subsection*{Using CCES data}

\subsection*{Using your own data}

\section{Summary and discussion} \label{sec:summary}

\subsection*{Next steps and cautions}

\section*{Acknowledgments}

\bibliography{references.bib}


\end{document}
